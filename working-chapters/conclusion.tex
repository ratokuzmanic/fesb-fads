\section{Zaključak}
Iako kompresija digitalnih fotografija ostvaruje dobre rezultate pri korištenju DCT-a kao osnovne transformacije, 
pri većim razinama kompresije kvaliteta se degradira zbog artefakata koji su posljedica blokovne segmentacije. 
S druge strane, podpojasno kodiranje bazirano na valovima i njegove implementacije u kompresijske algoritme, poput
JPEG 2000 pruža značajno unaprijeđenje u kvaliteti fotografije u niskoj dinamici bitova.

Usporedbom danih metoda transformacije je zaključeno da DWT pruža bolju mjeru kompresije i izbjegava problem blokovne
segmentacije i dozvoljava bolju lokalizaciji i u prostornoj i u frekvencijskoj domeni. Uspoređujući definirane MSE i
PSNR faktore DCT-a i DWT-a utvrđeno je da je DWT bolji od DCT-a ukoliko se uzme dovoljno veliki broj koeficijenata pri
visokoj mjeri kompresije.

Originalna mjeranja vršena od strane prvotnih autora su dana na slici.
