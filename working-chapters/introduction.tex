\section{Uvod}
Razvojem mrežnih, a naročito internet, tehnologija korištenje multimedijalnih sadržaja
je u kontinuiranom porastu. Jedan od najprisutnijih oblika multimedijalnog sadržaja su
slikovni sadržaji, odnosno digitalne fotografije.

Zbog sve veće upotrebe digitalne fotografije i češćih prijenosa istih putem mreža, javlja
se potreba za kompresijom zapisa digitalni fotografija kako bi se zahtjevi za mrežu smanjili
i omogućilo korisnicima da u što kraćem vremenu pristupe fotografijama.

U centru istraživanja kompresije digitalne fotografije egzistiraju pitanja o optimalnim 
prostorima boja, segmentaciji, metodama i algoritmima kvantizacije, kvantizacijskim tablicama,
entropijskom kodiranju i brojna druga pitanja. 

Autori izvornog rada \textit{Double Compression Of JPEG Image Using DWT Over RDWT} nude 
teoretski pregled metode rada kompresije, kao i pregled različitih metoda pretvorbe zapisa 
digitalne fotografije iz prostorne u frekvencijsku domenu, uspoređujući pritom DCT (engl. 
\textit{Discrete Cosine Transform}), DWT (engl. \textit{Discrete Wavelet Transform}) RDWT
(\textit{Redundant Wavelet Transform}) metode transformacije.

U ovom seminarskom radu ću prenijeti izlaganja autora kao i njihov značaj.
