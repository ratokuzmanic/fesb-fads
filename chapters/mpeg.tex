\section{MPEG kompresija}
S obzirom da se u steganalizi ovog rada koristi MP4 format koji je baziran na MPEG kompresiji, u ovom poglavlju
je dan osnovni pregled metode rada iste kao i objašnjenje važnosti svake vrste okvira po steganografiju.

\subsection{Metodologija rada}
MPEG je u osnovi kompresija s gubicima (engl. \textit{lossy compression}) koja iskorištava činjenicu da ljudski
vizualni sustav ne osjeća značajne promjene u visoko frekvencijskom području. Temelji se na ulaznim kontinuiranim slikama
u YCbCr ili YUV modelu boja koje segmentira, a potom putem DCT transformacije pretvara u frekvencijsko područje. Nakon toga
vrši kvantizaciju dobivenih vrijednosti, sukladno njihovoj važnosti i preddefiniranim kvantizacijskim tablicama koje su obično
različite za osvjetljenje i kromu.

\subsection{Vrste okvira i njihov steganografski potencijal}
MPEG kompresija koristi tri vrste okvira: I-, B- i P-okvire, kao što je prikazano na slici ~\ref{img:mpeg}.
\begin{figure}[h]
    \centering
    \includegraphics[width=0.58\textwidth]{mpeg.png}
    \caption{Primjer sekvence MPEG okvira}
    \label{img:mpeg}
\end{figure}
Tri vrste okvira i njihov steganografski potencijal je dan kako slijedi:
\begin{itemize}
    \item \textbf{I-okvir}: Intra-kodirani okviri su okviri koji mogu biti rekonstruirani bez upotrebe drugih okvira,
    odnosno samodostatni su. Svaki video započinje s jednim I-okvirom i periodično posjeduje nove I-okvire kako bi se
    spriječila propagacija pogreške u prediktivnom kodiranju. Njihov nedostatak u kodiranju je činjenica da su velike veličine
    s obzirom da je cijeli video kadar kodiran unutar okvira, no to im je steganografska prednost jer posjeduju najveći broj informacija,
    odnosno najveću entropiju koja je pogodna za omotače.
    \item \textbf{P-okvir}: Inter-kodirani P-okviri su okviri koji mogu biti rekonstruirani upotrebom prethodnih I- ili P-okvira,
    što znači da su osjetljivi na greške u prijenosu i generalno malene veličine i entropije (kao posljedica vremenske i prostorne kvantizacije)
    što ih čini lošim kandidatima za omotač.
    \item \textbf{B-okvir}: Inter-kodirani B-okviri su okviri koji mogu biti rekonstruirani upotrebom prethodnih i budućih I- ili P-okvira. Iako
    korištenje B-okvira poboljšava kvalitetu prediktivnog kodiranja i kvalitetu dekodiranog videa, istovremeno povećava vrijeme potrebno
    za procesiranje. Steganografski su analogni P-okvirima.    
\end{itemize}
I-okviri su najpodobniji za provođenje steganografskih metoda supstitucije.
