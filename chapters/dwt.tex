\section{DWT transformacija}
Kako bismo proveli DWT transformaciju, potrebno je učiniti sljedeće korake.

    \subsection{Dekompozicija}
    Dekompozicija se sastoji od sljedećih koraka:
    \begin{enumerate}
        \item Odabrati val i razinu N. Izračunati val i dekompozirati signal na razinu N.
        \item Sukscesivno filtrirati niskim (G) i visokim (H) filterom diskretni signal u vremenskoj domeni.
        \item Spremiti faktor d[n] koji je proizašao iz visokog filtera i sadrži detaljne informacije.
        \item Spremiti faktor a[n] koji je proizašao iz niskog filtera i sadrži informacije o grubim aproksimacijama.
        \item Svakom filtracijom smanjujemo širinu pojasa i povećajemo frekvenciju rezolucije.
    \end{enumerate}
    Maksimalni broj razina ovisi o duljini signala. Proces dekompozicije je oslikan sljedećim prikazom:
    \begin{figure}[h]
        \centering
        \includegraphics[width=0.9\textwidth]{dwt}
    \end{figure}

    \subsection{Određivanje granice koeficijenata detaljnih informacija}
    Za svaku razinu, od 1 do N, se odabire stroga granica za koeficijente detaljnih informacija.

    \subsection{Rekonstrukcija}
    Izračunati rekonstrukciju vala koristeći aproksimacijske koeficijente za nivo n
    i koeficijente detaljnih informacija od nivoa 1 do nivoa N.

    \subsection{Primjer}
Primjer fotografije na koju je primjenjena DWT transformacija u sklopu JPEG 2000 kodiranja
je dana uz originalnu sliku, respektivno, na sljedećoj slici:
\begin{figure}[h]
    \centering
    \includegraphics[width=0.42\textwidth]{original}
    \includegraphics[width=0.42\textwidth]{example-dwt}
\end{figure}
