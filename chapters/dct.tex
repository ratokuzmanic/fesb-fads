\section{DCT transformacija}
DCT transformacija je jedna od najčešćih transformacija koju primjenjujemo, a svoju popularnost naspram primjerice 
diskretne Furierove transformacije duguje sljedećim razlozima:
    \begin{enumerate}
        \item Ljudsko oko je senzitivnije na niske frekvencije nego li visoke.
        \item DCT može smanjiti utjecaj kvadrataste segmentacije.
    \end{enumerate}

Matematički gledano, DCT transformacija se definira kako slijedi:
F(u,v) = $\frac{2}{N}*C(u)*C(v)\sum_{x=0}^{N-1}\sum_{y=0}^{N-1}f(x,y)cos(\frac{\pi*(2x+1)*u}{2*N})cos(\frac{\pi*(2y+1)*v}{2*N})$
za u = 0, ..., N - 1 i v = 0, ..., N - 1
gdje N = 8 i C(k) = $\frac{1}{\sqrt{2}}$ za k = 0 i 1 inače.

    \subsection{Implementacija}