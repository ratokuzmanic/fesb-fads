\section{Arhitektura EME sustava}
Na slici Slika 1. možemo vidjeti referentnu arhitekturu DRM sustava opisanog EME
standardom. [1] Arhitektura se sastoji od sljedećih komponenti:
\begin{enumerate}
\item Pristupne točke krajnjem OTT servisu putem koje se dohvaća multimedijalni sadržaj,
najčešće ostvaren putem CDN-a (engl. Content Delivery Network) kako bi se pospješila
distribuiranost krajnjih servera i omogućila niska latencija kao i pravilno geoprostorno
opterećenje.
\item Preglednika kojeg krajnji korisnik upotrebljava kako bi pristupio multimedijalnom
sadržaju na ciljanom OTT servisu putem internetske infrastrukture. Korišteni
preglednik je predmet odabira krajnjeg korisnika, no izdavatelj licenci može ograničiti
dozvoljene preglednike s ciljem dodatne zaštite u slučaju pronalaska propusta u
implementaciji EME sustava u nekom od preglednika.
\item Aplikacije koja se odnosi na softverski sustav pružen krajnjem korisniku, a omogućava
mu dohvat potrebnih komponenti za upotrebu DRM sustava i dekripciju zaštićenog,
najčešće kriptiranog, multimedijalnog sadržaja.
\item Web servera koja predstavlja OTT servis u širem smislu i omogućava korisniku da
putem sigurnog kanala autorizira svoje radnje, poglavito prikupljanjem nužnih licenci
za upotrebu multimedijalnog sadržaja.
\item CDM (engl. \textit{Content Decryption Module}) je softverski sustav koji može biti neovisan
o samome pregledniku, no mora pružati mogućnost dekripcije zaštićenog
multimedijalnog sadržaja ukoliko je isti u ispravnom formatu i pružena je
odgovarajuća, validna licenca.
\item Multimedijalni okvir koji upravlja komunikacijom između aplikacije i CDM sustava
putem preglednika. Zadaća ovog okvira je na siguran način upravljati licencama kao i
nadolazećim segmentima multimedijalnog sadržaja koje mu osigurava preglednik.
\item Platforma koja ne mora odgovarati ugrađenoj video platformi preglednika, iako
referentna arhitektura predlaže korištenje iste kako bi se smanjio tehnološki dug kao i
povećala portabilnost rješenja.

SLIKA

\subsection{Lanac povjerenja}
Prilikom dizajniranja arhitekture sustava, uzete su sljedeće pretpostavke o lancu povjerenja u
obzir [2]:
\begin{enumerate}
\item Izdavatelj licence ne vjeruje pregledniku.
\item Izdavatelj licence vjeruje CDM-u.
\item Izdavatelj licence vjeruje web serveru.
\item Izdavatelj licence vjeruje multimedijalnom okviru i platformi dokle god je ista sukladna
EME standardu.

Ovakvim lancem povjerenja, izdavatelji licence su zadržali isključivo pravo odabira platforme
i mehanizma dekripcije, tj. CDM-a, kojeg mogu pružiti u vidu vlastitog softverskog rješenja koje
korisnik primjerice može preuzeti u vlastiti preglednik, instalirati lokalno na računalno, imati
integriranog s operacijskim sustavom ili pak ugrađenog u sami hardver. Uvođenjem ove restrikcije 
povjerljivost multimedijalnog sadržaja nije narušena sve do trenutka dok CDM ne dekriptira
sadržaj i proslijedi istog reproducirajućoj platformi.

U praksi se obično integriraju na tržištu testirana rješenja za upravljanje ključevima i
licencama, bilo da je riječ o Microsoftovom PlayReady rješenju ili Googleovom Widevine.
Usprkos činjenici da se isti još uvijek koriste kao sastavni dio upravljanja ključeva u CDM-ovima,
njihov tehnološki doseg i ovisnost je smanjena u vidu apstrahiranja samog procesa upravljanja
DRM sustavom i reprodukcijom multimedijalnog sadržaja. [2]

\subsection{Metodologija rada}
Sami EME standard ne specificira razinu apstrakcije koju CDM mora implementirati, pa je
stoga moguće da sustav nakon dekripcije:
\begin{enumerate}
\item  Pruži dekriptirani multimedijalni sadržaj u kodiranom obliku (npr. JPEG, H.264, MPEG4, itd.) pregledniku ili platformi;
\item  Dekriptira multimedijalni sadržaj koji potom dekodira i pruži okvire za prikaz pregledniku
ili platformi;
\item  Dekriptira multimedijalni sadržaj, dekodira ga i potom pruži informacije o pikselima koje
je potrebno iscrtati u pregledniku ili platformi;
\item  Dekriptira multimedijalni sadržaj, dekodira ga i potom pruži informacije o pikselima koje
je potrebno iscrtati direktno grafičkoj kartici (hardveru).
\end{enumerate}
Ovisno o granulaciji apstrakcije i razini kontrole pristupa dekodiranim podacima i okvirima
definira se rigoroznost samog CDM sustava kao i vjerojatnost da izdavatelji licenca pristanu na
korištenje istog. [2]
Primjer korištenja DRM sustava koji se bazira na EME standardu i koristi vlastiti CDM je dan
kako slijedi:
\begin{enumerate}
\item Preglednik dohvaća multimedijalni sadržaj sa pristupne točke koristeći pritom proizvoljni
protokol za prijenos sadržaja. Sami standard ne definira nužan protokol ili svojstva
transmisije, iako se u praksi najčešće koriste protokoli za kontinuirani prijenos
multimedijalnog sadržaja, primjerice HLS (engl. \textit{HTTP Live Streaming}) ili MPEG-DASH
(engl. \textit{Dynamic Adaptive Streaming over HTTP}).
\item Kada preglednik primi kriptirani multimedijalni sadržaj, okida događaj kojim obavještava
multimedijalni okvir koji se izvršava u kontekstu stranice da je preglednik naišao na
multimedijalni (najčešće video) sadržaj za čije je očitanje potreban ključ. Zajedno s tim,
preglednik također prosljeđuje inicijalizacijske podatke iz multimedijalnog sadržaja
samome multimedijalnom okviru.
\item  Multimedijalni okvir kreira sesiju i prosljeđuje podatke o inicijalizacijskim podacima
CDM-u koji korištenjem svog sustava za upravljanje ključevima kreira sesijski ključ.
Pritom ukoliko postoji više implementiranih sustava za upravljanje ključem (npr. za
podržavanje više platformi), multimedijalni okvir može izabrati koji sustav želi koristiti.
\item Multimedijalni okvir pokrene transmisiju okteta multimedijalnog sadržaja kodiranog u
format koji pruži CDM i njegov odgovarajući sustav za upravljanje ključevima. Preneseni
okviri osiguravaju povjerljivost jer niti preglednik niti multimedijalni okvir ne znaju
dekriptirati iste.
\item Podaci o aktivnoj sesiji se prenose iz multimedijalnog okvira u CDM koji korištenjem te
informacije, primljenih okteta multimedijalnog sadržaja i generiranog sesijskog ključa
kreira kriptografski dokaz o vlastitoj autentičnosti i ispravnosti. Taj dokaz se prosljeđuje
multimedijalnom okviru.
\item Multimedijalni okvir prenosi kriptografski dokaz, zajedno uz informacije o sesiji i
korisniku, web serveru koji pritom konstruira poruku koja sadrži ključ potreban CDM-u da
bi ispravno dekriptirao asocirani multimedijalni sadržaj. Web server šalje tu poruku
multimedijalnom okviru.
\item Multimedijalni okvir koji, kao i preglednik, ne može narušiti povjerljivost poruke koju je
dobio od web servera istu prosljeđuje CDM-u.
\item CDM dobivene ključeve koristi kako bi dekriptirao multimedijalni sadržaj i istog
proslijedio pregledniku ili platformi, ovisno o implementaciji.