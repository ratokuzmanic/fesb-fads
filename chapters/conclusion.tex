\section{Zaključak}
Provođenjem steganalize nad prikupljenim materijalima s HBO GO platforme, pokazano je da postojanje
steganografski informacija unutar video sadržaja nije vjerojatno. Pritom postoji osnovica za daljnji razvoj
i istraživanja ove teme, od same metodologije rada pa do povećanja broja testiranih video sadržaja, kao i OTT
platformi.

U vidu proširenja metoda rada i povećanja intervala sigurnosti u dani zaključak, može se provesti frekvencijska
analiza inter okvira samih video materijala, kao i kompleksnija analiza supstitucije koja dozvoljava pseudo
nasumično umetanje tajne poruke. Također, primjenjene metode bi bile od većeg značaja ako se primjene na statistički
signifikantniji skup, što otvara mogućnost kreiranja aplikacije za automatizirano segmentiranje, ekstrahiranje i analiziranje
okvira.

S obzirom na neočekivana otkrića u pogledu različitosti kadriranja verzija video sadržaja u ovisnosti o reproducirajućoj
platformi, budući radovi se mogu orijentirati i testiranju broja dijeljenih kadrova među različitim verzijama videa, bilo
onima posluženim na drugim platformama ili drugim korisnicima. Također, s obzirom na adaptivnu prirodu toka bitova unutar 
OTT servisa, od interesa bi moglo biti i analiziranje promjena brzine bitova u kodiranju s obzirom na statičke uvjete i 
dinamične promjene korisnika i sesija.

Iako ovaj rad pruža osnovni teoretski pregled eksperimenta, kao i jednostavnu analizu, vidimo da postoji puno prostora za 
usavršavanje i daljnji razvoj ovog istraživanja i analize.
