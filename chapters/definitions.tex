\section{Definitions}
Throughout the specification a following list of terms is widely used. In this section their meaning and assumptions 
made on them are given in details.

\medskip
\textbf{Email}: Email client that allows the user to gain read access rights to the inbox associated
with the email address they provided. It is assumed that the user is the only person with this privilege.

\medskip
\textbf{Auth}: Central authentication provider that serves as a secure token store (STS) and conforms to 
the Raccu Auth Provider API Specification. Valid and trusted certificate issued by a certificate authority
(CA) as a part of PKI for this entity is implied.

\medskip
\textbf{Mobile}: An implementation of the mobile application that communicates with the corresponding Auth instance 
conforming to the Raccu Mobile Application Specification. Ability to authenticate the user via a fingerprint scanner
on the host smartphone is implied.

\medskip
\textbf{Browser}: An arbitrary web browser in charge of rendering quick response (QR) codes and dispatching appropriate 
messages as specified by the Raccu Client Component Specification.

\medskip
\textbf{Attestation}: Digitally signed claim in a form of a token that confirms holder's identity. Issued by 
Auth in a format defined in this specification.

\medskip
\textbf{Server}: An end web service or application that consumes an attestation. Authorization service within the 
Server is assumed. Valid and trusted certificate issued as a part of PKI for this entity is implied.

\medskip
The key words "MUST", "MUST NOT", "REQUIRED", "SHALL", "SHALL NOT", "SHOULD", "SHOULD NOT", "RECOMMENDED", 
"NOT RECOMMENDED", "MAY", and "OPTIONAL" in this document are to be interpreted as described in 
\href{https://tools.ietf.org/html/rfc2119}{RFC 2119} and \href{https://tools.ietf.org/html/rfc8174}{RFC 8174} 
when, and only when, they appear in all capitals, as shown here.
