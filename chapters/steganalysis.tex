\section{Steganaliza}
Steganaliza je znanost otkrivanja steganografski skrivenih informacija, analogno kriptoanalizi u kriptografiji.
U pravilu se provodi aktivna ili pasivna steganaliza, gdje u aktivnoj analizi analitičar može manipulirati podacima,
a u pasivnoj ih isključivo analizirati. 

\subsection{Tipovi napada}
S obzirom na uvjete provođenja analize s ciljem otkrivanja steganografije, koji se još zove i \textit{napadom},
sljedeći tipovi napada su definirani: \cite{steganalysis}
\begin{itemize}
    \item \textbf{Samo stego napad}: Analitičaru su poznati samo steganografski modificirani podaci, no analitičar ne zna
    jesu li isti modificirani ili izvorni podaci.
    \item \textbf{Poznat omotač napad}: Analitičaru su poznati steganografski modificirani podaci i izvorni omotač koji je korišten
    za dobijanje steganografski modificiranih podataka.
    \item \textbf{Poznata skrivena informacija napad}: Analitičaru su poznati steganografski modificirani podaci, izvorni omotač koji 
    je korišten za dobijanje steganografski modificiranih podataka i skrivena informacija.
    \item \textbf{Izabrani stego napad}: Analitičaru su poznati alati i metodologije koje su korištene pri izradi steganografski modificiranih
    podataka.
    \item \textbf{Izabrana skrivena informacija napad}: Analitičaru može generirati proizvoljne steganografski modificirane podatke iz izabranog
    omotača i skrivene informacije.
\end{itemize}

\subsection{Metode napada}
Metode napada razlikujemo temeljem fundamentalnih pristupa koji isti zauzimaju u provođenju napada.
    \subsubsection{Vizualni napad}
    Vizualni napad je primjer najjednostavnijeg napada na steganografski sustav. Analitičar golim okom pokušava uočiti
    razlike između omotača i steganografski modificiranih podataka, a ukoliko iste ne uvidi ili ukoliko mu izvorni omotač nije poznat,
    tj. ako je riječ o \textit{samo stego napadu} tada analitičar analizira ravnine slike (engl. \textit{bit plane slicing}), počevši s
    ravninom najmanje značajnog bita.
    \begin{figure}[h]
        \caption{Primjer vizualnog napada}
        \centering
        \includegraphics[width=0.74\textwidth]{visual-attack}
        \label{img:visualAttack}
    \end{figure}
    \subsubsection{Statistički napad}
    Statističkim napadom se ispituju matematička svojstva podataka i temeljem njihovih analiza saznati postoji li skrivena
    informacija ili ne. Statistički napadi su u pravilu efektivniji od vizualnih napada i najbolji pristup u \textit{samo stego napadu}.
    
    Jedan od najpoznatijih statističkih napada je Chi-square napad. \cite{chisquare} Chi-square napad je mjera sličnosti očekivanih podataka
    i promatranih podataka. U stegoanalizi, ovaj napad pruža uvid u vjerojatnost postajanja skrivene informacije. Napad se može uspješno provesti
    čak i bez poznavanja omotača, no ne može razlučiti veličinu skrivene informacije.
