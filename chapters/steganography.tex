\section{Steganografija}
Iako u ovom radu promatramo steganografiju na primjeru video sadržaja, temeljni koncept je nepromijenjen u usporedbi
sa steganografijom digitalne fotografije ili audio zapisa.

\subsection{Definicija}
Steganografija (grčki \textit{steganos} + \textit{graphein}, prikriveno pisanje) je znanost tajne komunikacije, odnosno tehnika prikrivanja informacije u omotač tako da nitko, osim
krajnje namjenjenog primatelja, ne zna za samo postojanje prikrivene informacije. Uobičajeni steganografski sustav
je dan na slici ~\ref{img:steganography}.
\begin{figure}[h]
    \caption{Uobičajeni steganografski sustav}
    \centering
    \includegraphics[width=0.9\textwidth]{steganography.png}
    \label{img:steganography}
\end{figure}

\subsection{Usporedba s vodenim žigom}
Vodeni žig je sličan steganografiji, no pritom je cilj steganografije da prikrivena informacija ostane nedetektirana,
dok vodeni žig može biti i evidentan dokle god njegovo uklanjanje uzrokuje značajnu degradaciju kvalitete nositelja. 
Vodeni žig se često koristi kako bi se pripomoglo praćenje izvora fotografije.

\subsection{Usporedba s kriptografijom}
Glavna razlika između steganografije i kriptografije jest u tome što kriptografija kreira poruku koja nije razumljiva,
dok steganografija prikriva poruku tako da ne bude vidljiva. I kriptografija i steganografija imaju za cilj zaštiti
povjerljivost informacije.

\subsection{Poveznica s DRM sustavima}
DRM sustavi primjenjuju kriptografske metode, kao što je prikazano u prethodnom poglavlju na konkretnom EME standardu,
kako bi zaštitili povjerljivost multimedijalnog sadržaja. Iako svojim mehanizmima rada ne povisuju ljestvicu povjerljivosti,
steganografske metodologije pružaju dodatnu razinu zaštite u pogledu mogućnosti praćenja izvora informacije, tj. integriteta izvora.

Korištenjem steganografije u DRM sustavu možemo pratiti svaku korisničku sesiju i automatski saznati koji korisnik je narušio licencom
dana prava. \cite{newWatermark} Kao što je prikazano u prethodnom poglavlju, EME bazirani sustavi već koriste pojam \textit{sesije}, a iste
podatke dijele i s autorizacijskim i licencnim serverom zbog čega je enkodiranje sadržaja sa ugrađenim steganografskim dokazima
porijekla održiva opcija.

\begin{figure}[h]
    \caption{VideoMark steganografski identifikator}
    \centering
    \includegraphics[width=0.65\textwidth]{videomark}
    \label{img:videomark}
\end{figure}

\subsection{Metode rada}
Steganografija se implementira brojnim metodama rada, od kojih su najpoznatije navedene u ovom odlomku.
    \subsubsection{Injekcija}
    Najjednostavnija metoda koja se zaniva na umetanju tajne poruke u dijelu datoteke koji će biti zanemaren
    od strane klijentskog programa. Ova metoda se može implementirati, primjerice, umetanjem skrivene poruke
    u komentare, skrivene elemente ili nakon oznaka za kraj datoteke (EOF -- engl. \textit{End of File}). Glavni
    nedostatak ove metode jest da povećava veličinu modificirane datoteke u usporedbi s nemodificiranom.

    \subsubsection{Supstitucija}
    Supstitucijske metode rada se zasnivaju na identifikaciji dijela datoteke s najmanjom važnoću i zamjeni istog
    sa skrivenom porukom. Ova metoda ne modificira veličinu datoteke, no stoga je ograničenog kapaciteta koji ovisi
    o mogućnostima degradacije izvorne datoteke i entropiji iste.
        \subsubsubsection{LSB supstitucija}
        Jedna od najčešćih implementacija supstitucije je primjenom zamjene najmanje značajnog bita (LSB -- engl. \textit{Least Significant Bit}).
        \cite{mostCommonLSB}

        Primjerice, u 8 bitnim slikama, svaki piksel je predstavljen s 8 bita. Na slici ~\ref{img:lsbSupstitucija} je prikazan primjer takve
        slike gdje se na krajnjem lijevom mjestu nalazi najznačajniji bit (MSB -- engl. \textit{Most Significant Bit}), a 
        na krajnje desnom mjestu najmanje značajan bit (LSB). Promjenom MSB značajno mijenjamo izvornu sliku, no promjenom
        LSB ostvarujemo teško primjetnu razliku, dokle god je slika koju mijenjamo dobar kandidat za omotač (ima visoku
        entropiju).
        \begin{figure}[h]
            \caption{Prikaz rada LSB supstitucije}
            \centering
            \includegraphics[width=0.9\textwidth]{lsb.png}
            \label{img:lsbSupstitucija}
        \end{figure}

        Pritom kako bi se osigurao integritet skrivene informacije se najčešće koriste kompresije bez gubitka podataka (engl. \textit{lossless compression}).
        Glavna prednost ove metode je lakoća njene implementacije, no stoga je podložna ekstrakciji i napadima kompresijom ili izrezivanjem.

        S obzirom na metodu supstitucije, LSB supstituciju dijelimo na sljedeće oblike:
        \begin{itemize}
            \item \textbf{Sekvencijalna}: U sekvencijalnoj supstituciji, kao što ime nalaže, skrivena informacija se pohranjuje
            sekvencijalno, tj. u kontinuiranom slijedu podataka omotača. Ova tehnika je jednostavnija za detektiranje i dekodiranje s
            obzirom da promjena u bitu prati sljedeću promjenu.
            \item \textbf{Pseudo nasumična}: U pseudo nasumičnoj supstituciji se koristi pseudo nasumični generator brojeva (PSRNG -- engl. \textit{Pseudo Random Number Generator})
            koji na nasumičan način određuje lokacije pohrane bitova skrivene informacije unutar omotača. Preduvjet ove metode jest dijeljena tajna
            koja se koristi kao inicijalizator generatora. Ova tehnika je značajno teža za detekciju i statistički uniformnija.
        \end{itemize}

    \subsubsection{Transformacije}
    Postoje i kompliciranije metode skrivanja informacije, poput modifikacije frekvencijskog područja primjenom diskretne
    kosinusne (DCT -- engl. \textit{Discrete Cosine Transform}) ili valićne transformacije (DWT -- engl. \textit{Discrete Wavelet Transform}), no one su
    van domene ovog rada. 
