\section{Steganografija}
Iako u ovom radu promatramo steganografiju na primjeru video sadržaja, temeljni koncept je nepromijenjen u usporedbi
sa steganografijom digitalne fotografije ili audio zapisa.

\subsection{Definicija}
Steganografija (grčki \textit{steganos} + \textit{graphein}, prikriveno pisanje) je znanost tajne komunikacije, odnosno tehnika prikrivanja informacije u omotač tako da nitko, osim
krajnje namjenjenog primatelja, ne zna za samo postojanje prikrivene informacije. Uobičajeni steganografski sustav
je dan na slici ~\ref{img:steganography}.
\begin{figure}[h]
    \caption{Uobičajeni steganografski sustav}
    \centering
    \includegraphics[width=0.9\textwidth]{steganography.png}
    \label{img:steganography}
\end{figure}

\subsection{Usporedba s vodenim žigom}
Vodeni žig je sličan steganografiji, no pritom je cilj steganografije da prikrivena informacija ostane nedetektirana,
dok vodeni žig može biti i evidentan dokle god njegovo uklanjanje uzrokuje značajnu degradaciju kvalitete nositelja. 
Vodeni žig se često koristi kako bi se pripomoglo praćenje izvora fotografije.

\subsection{Usporedba s kriptografijom}
Glavna razlika između steganografije i kriptografije jest u tome što kriptografija kreira poruku koja nije razumljiva,
dok steganografija prikriva poruku tako da ne bude vidljiva. I kriptografija i steganografija imaju za cilj zaštiti
povjerljivost informacije.

\subsection{Poveznica s DRM sustavima}
DRM sustavi primjenjuju kriptografske metode, kao što je prikazano u prethodnom poglavlju na konkretnom EME standardu,
kako bi zaštitili povjerljivost multimedijalnog sadržaja. Iako svojim mehanizmima rada ne povisuju ljestvicu povjerljivosti,
steganografske metodologije pružaju dodatnu razinu zaštite u pogledu mogućnosti praćenja izvora informacije, tj. integriteta izvora.

Korištenjem steganografije u DRM sustavu možemo pratiti svaku korisničku sesiju i automatski saznati koji korisnik je narušio licencom
dana prava. \cite{newWatermark} Kao što je prikazano u prethodnom poglavlju, EME bazirani sustavi već koriste pojam \textit{sesije}, a iste
podatke dijele i s autorizacijskim i licencnim serverom zbog čega je enkodiranje sadržaja sa ugrađenim steganografskim dokazima
porijekla održiva opcija.

\begin{figure}[h]
    \caption{VideoMark steganografski identifikator}
    \centering
    \includegraphics[width=0.65\textwidth]{videomark}
    \label{img:videomark}
\end{figure}
