\section{Reset credentials}
In this section, a flow for resetting credentials in case a user loses their smartphone, they no longer have access to it,
or they wish to reset credentials for any other reason is defined.
\medskip

The reset credentials flow is given in Figure~\ref{fig:resetCredentials} and a successful flow is defined as follows:
    \begin{enumerate}
        \item Mobile MUST request a verification code from the Auth by submitting the email address of the user
              which is seeking a credentials reset.
        \item Upon receiving a reset credentials request, the Auth MUST dispatch an email with the verification
              code to the Email and a token MUST be returned to the Mobile.
        \item When the user enters verification code into the Mobile, the Mobile MUST provide a method to generate,
              store, and retrieve public key and an interface for signing an arbitrary message with the matching 
              private key.
        \item Mobile MUST authenticate the user locally using a built-in biometry. Upon authenticating, Mobile 
              MUST send the token, user's email, newly generated public key, and a digital signature of all of the 
              previous parameters, with a REQUIRED addition of the verification code, to the Auth to finish the 
              credentials reset.
    \end{enumerate}
    \section{Reset credentials}
In this section, a flow for resetting credentials in case a user loses their smartphone, they no longer have access to it,
or they wish to reset credentials for any other reason is defined.
\medskip

The reset credentials flow is given in Figure~\ref{fig:resetCredentials} and a successful flow is defined as follows:
    \begin{enumerate}
        \item Mobile MUST request a verification code from the Auth by submitting the email address of the user
              which is seeking a credentials reset.
        \item Upon receiving a reset credentials request, the Auth MUST dispatch an email with the verification
              code to the Email and a token MUST be returned to the Mobile.
        \item When the user enters verification code into the Mobile, the Mobile MUST provide a method to generate,
              store, and retrieve public key and an interface for signing an arbitrary message with the matching 
              private key.
        \item Mobile MUST authenticate the user locally using a built-in biometry. Upon authenticating, Mobile 
              MUST send the token, user's email, newly generated public key, and a digital signature of all of the 
              previous parameters, with a REQUIRED addition of the verification code, to the Auth to finish the 
              credentials reset.
    \end{enumerate}
    \section{Reset credentials}
In this section, a flow for resetting credentials in case a user loses their smartphone, they no longer have access to it,
or they wish to reset credentials for any other reason is defined.
\medskip

The reset credentials flow is given in Figure~\ref{fig:resetCredentials} and a successful flow is defined as follows:
    \begin{enumerate}
        \item Mobile MUST request a verification code from the Auth by submitting the email address of the user
              which is seeking a credentials reset.
        \item Upon receiving a reset credentials request, the Auth MUST dispatch an email with the verification
              code to the Email and a token MUST be returned to the Mobile.
        \item When the user enters verification code into the Mobile, the Mobile MUST provide a method to generate,
              store, and retrieve public key and an interface for signing an arbitrary message with the matching 
              private key.
        \item Mobile MUST authenticate the user locally using a built-in biometry. Upon authenticating, Mobile 
              MUST send the token, user's email, newly generated public key, and a digital signature of all of the 
              previous parameters, with a REQUIRED addition of the verification code, to the Auth to finish the 
              credentials reset.
    \end{enumerate}
    \section{Reset credentials}
In this section, a flow for resetting credentials in case a user loses their smartphone, they no longer have access to it,
or they wish to reset credentials for any other reason is defined.
\medskip

The reset credentials flow is given in Figure~\ref{fig:resetCredentials} and a successful flow is defined as follows:
    \begin{enumerate}
        \item Mobile MUST request a verification code from the Auth by submitting the email address of the user
              which is seeking a credentials reset.
        \item Upon receiving a reset credentials request, the Auth MUST dispatch an email with the verification
              code to the Email and a token MUST be returned to the Mobile.
        \item When the user enters verification code into the Mobile, the Mobile MUST provide a method to generate,
              store, and retrieve public key and an interface for signing an arbitrary message with the matching 
              private key.
        \item Mobile MUST authenticate the user locally using a built-in biometry. Upon authenticating, Mobile 
              MUST send the token, user's email, newly generated public key, and a digital signature of all of the 
              previous parameters, with a REQUIRED addition of the verification code, to the Auth to finish the 
              credentials reset.
    \end{enumerate}
    \input{diagrams/sequence/reset-credentials.tex}
If the credentials reset fails for any reason related to the protocol, an attempt to reset credentials MUST 
be discarded. All of the following reasons are considered to be related to the protocol:
    \begin{itemize}
        \item An email is not associated with any account.
        \item Token does not exist.
        \item Token is not associated with a given email.
        \item Verification code entered into Mobile does not match the expected verification code.
        \item The signature cannot be verified by a given public key.
        \item The signature is invalid.
    \end{itemize}

If the credentials reset fails for any reason related to the protocol, an attempt to reset credentials MUST 
be discarded. All of the following reasons are considered to be related to the protocol:
    \begin{itemize}
        \item An email is not associated with any account.
        \item Token does not exist.
        \item Token is not associated with a given email.
        \item Verification code entered into Mobile does not match the expected verification code.
        \item The signature cannot be verified by a given public key.
        \item The signature is invalid.
    \end{itemize}

If the credentials reset fails for any reason related to the protocol, an attempt to reset credentials MUST 
be discarded. All of the following reasons are considered to be related to the protocol:
    \begin{itemize}
        \item An email is not associated with any account.
        \item Token does not exist.
        \item Token is not associated with a given email.
        \item Verification code entered into Mobile does not match the expected verification code.
        \item The signature cannot be verified by a given public key.
        \item The signature is invalid.
    \end{itemize}

If the credentials reset fails for any reason related to the protocol, an attempt to reset credentials MUST 
be discarded. All of the following reasons are considered to be related to the protocol:
    \begin{itemize}
        \item An email is not associated with any account.
        \item Token does not exist.
        \item Token is not associated with a given email.
        \item Verification code entered into Mobile does not match the expected verification code.
        \item The signature cannot be verified by a given public key.
        \item The signature is invalid.
    \end{itemize}
