\section{Teorijske osnove kompresije}
Pohranjivanje digitalne fotografije podrazumijeva proces pretvorbe izvorne slike u niz bitova
(engl. \textit{bitstream}) korištenjem kodera, te potom očitavanje pretvorbom niza bitova u
izvornu sliku korištenjem dekodera. Ukoliko je količina podataka sadržana u kodiranom obliku
izvorne fotografije manje nego li količina podataka u izvornoj fotografiji, tada taj proces
nazivamo kompresijom. Slijed ovog procesa je prikazan slikom BROJ SLIKE.

Omjer kompresije je dan izrazom: 
$Omjer kompresije = \frac{n_{1}}{n_{2}}$
gdje $n_{1}$ predstavlja količinu podataka u izvornoj fotografiji, a $n_{2}$ količinu podataka u
nizu bitova. 

    \subsection{Mjerenje kvalitete kompresije}
    U svrhe mjerenja kvalitete kompresije, poglavito koristimo dvije metode:
    \begin{enumerate}
        \item \textbf{MSE} (engl. \textit{Mean Square Error}) koji je definiran na sljedeći način:
        $MSE = \sqrt{\frac{\sum_{x=0}^{W-1}\sum_{y=0}^{H-1}[f(x,y) - {f}'(x,y)]^{2}}{W*H}}$
        \item \textbf{PSNR} (engl. \textit{Peak Signal to Noise Ratio}) koji je definiran na sljedeći 
        način: $PSNR = 20*log_{10}\frac{255}{MSE}$
    \end{enumerate}
    Gdje pritom f(x,y) predstavlja vrijednost piksela u izvornoj fotografiji, a {f}'(x,y) predstavlja
    vrijednost piksela dekodirane slike.

    \subsection{Vrste transformacija}
    U pogledu provođenja transformacije izvorne fotografije iz prostorne u frekvencijsku
    domenu, najčešće koristimo neku od metoda iz jedne od sljedećih kategorija:
    \begin{enumerate}
        \item \textbf{Prediktivno kodiranje}: Jedan od primjera prediktivnog kodiranja je 
        DPCM (engl. \textit{Differential Pulse Code Modulation}), a zasniva se na kompresiji
        bez gubitaka, odnosno komprimirana fotografija ima istu vrijednost kao izvorna za svaki
        odgovarajući element.
        \item \textbf{Ortogonalna transformacija}: Jedna od najpoznatijih ortogonalnih transformacija
        je DCT (engl. \textit{Discrete Cosine Transform}) koji se koristi i u JPEG standardu koji
        komprimira slike s gubitkom kvalitete.
        \item \textbf{Podpojasno kodiranje}: Podpojasno kodiranje poput DWT (engl. \textit{Discrete 
        Wavelet Transform}) također komprimira slike s gubitkom kvalitete. Cilj podpojasnog kodiranja
        je podijeliti spektar slike na niskopojasnu i visokopojasnu komponentu. JPEG 2000 je DWT baziran
        standard koji koristi dvodimenzionalnu kompresiju.
    \end{enumerate}

