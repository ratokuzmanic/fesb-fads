\section{Uvod}
Porastom popularnosti OTT (engl. \textit{Over the Top}) multimedijalnih servisa poput 
Netflixa, Hulua, HBO GO-a, Amazon Videa i drugih, nameće se pitanje tehnološke zaštite 
intelektualnog vlasništva i prava na autorsko djelo. U širem smislu, vlasnici autorskih 
prava žele zadržati pravo kontrole toka, upotrebe i distribucije svojih djela.

Razvoj OTT servisa je pratio razvoj DRM (engl. \textit{Digital Rights Management}) sustava
koji osiguravaju kontrolu pristupa multimedijalnom sadržaju i temeljem kojeg se određenim
korisnicima, odnosno konzumentima, daje isključivo pravo na korištenje medija u unaprijed
definiranom obliku. Moderni DRM sustavi namijenjeni OTT servisima većinom koriste W3C-ov EME
(engl. \textit{Encrypted Media Extensions}) standard koji omogućava jednostavnu integraciju 
DRM rješenja u web preglednike klijenata bez narušavanja lanca povjerenja i uvođenja 
tehnoloških ovisnosti i tehnološkog višaka.

Prvenstvena namjena DRM sustava je zaštita povjerljivosti krajnjeg digitalnog zapisa,
no ista ne osigurava potpunu zaštitu povjerljivosti samog sadržaja koji je podložan 
primitivnim oblicima napada poput presnimavanja. Kako bi se omogućilo određivanje identiteta
inicijalnog pirata samog sadržaja, u sustav zaštite se uvode i steganografske metodologije i
tehnike skrivenog vodenog žiga, gdje se na korisniku neprimjetan način uvodi jedinstveni 
identifikator sesije.

Cilj ovog seminarskog rada je prikazati najčešće metode steganalize i na praktičnom primjeru 
serije t@gged s HBO GO platforme provesti stegoanalizu i pokušati pronaći dokaze identificirajućih 
podataka pohranjenih u sami video zapis.
