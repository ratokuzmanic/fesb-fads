\section{Introduction}
This document provides a formal specification of the Raccu protocol. All readers are assumed to have a working 
knowledge in cryptography and computer networking. As this is a high-level overview, implementation details are 
omitted intentionally.

    \subsection{Background}
    Raccu is a centralized secure token service (STS) protocol for public-key cryptography authentication 
    based on a smartphone biometry. The protocol is architecturally inspired by the likes of OAuth 2.0, OpenID, 
    and W3C's WebAuthn while it conceptually resembles FIDO Alliance's Client-to-Authenticator Protocol (CTAP), 
    namely FIDO UAF standards.

    \subsection{Goals}
    The core vision of Raccu is to provide a user with a uniform method of authentication possessing the 
    following properties: 
        \begin{enumerate}
            \item Smaller complexity and lesser cognitive load than in a traditional email and password 
                  authentication system.
            \item Single registration process for authenticating to multiple web services and applications.
            \item Authentication process for issuing an attestation is independent of the type of registration 
                  used to create an account.
            \item Preserving data and origin integrity of the attestation through public key infrastructure (PKI) 
                  based chain of trust.
            \item Requiring user's consent via independent biometric authorization process before granting an 
                  attestation.
        \end{enumerate}

     \subsection{Contributing}
     Anyone can contribute to the protocol or to its specification. To learn more about contributing, please visit our 
     official repository on GitHub located at \href{https://www.github.com/Raccu/Documentation}{https://www.github.com/Raccu/Documentation}.

    \subsection{Disclaimer}
    Implementation of certain elements of this specification may require licenses under third party intellectual 
    property rights, including without limitation, patent rights. The authors and any other contributors to this 
    specification are not, and shall not be held, responsible in any manner for identifying or failing to identify 
    any or all such third party intellectual property rights.    
    
    \medskip
    THIS SPECIFICATION IS PROVIDED “AS IS” AND WITHOUT ANY WARRANTY OF ANY KIND, INCLUDING, WITHOUT LIMITATION,
    ANY EXPRESS OR IMPLIED WARRANTY OF NON-INFRINGEMENT, MERCHANTABILITY OR FITNESS FOR A PARTICULAR PURPOSE.
